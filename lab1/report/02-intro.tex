
\chapter{Содержание}

\textit{Цель работы}: построение гистограммы и эмпирической функции распределения.
\begin{itemize}
	\item Для выборки объема n из генеральной совокупности X реализовать в виде программы на ЭВМ:
	\begin{enumerate}
		\item вычисление максимального значения $M_{max}$ и минимального значения $M_{min}$;
		\item размаха $R$ выборки;
		\item вычисление оценок $\hat{\mu}$ и $S^2$ математического ожидания MX и дисперсии DX;
		\item группировку значений выборки в $m = [\log_2n] + 2$ интервала;
		\item построение на одной координатной плоскости гистограммы и графика функции плотности распределения вероятностей нормальной случайной величины с математическим ожиданием $\hat{\mu}$ и дисперсией $S^2$;
		\item построение на другой координатной плоскости графика эмпирической функции распределения и функции распределения нормальной случайной величины с математическим ожиданием $\hat{\mu}$ и дисперсией $S^2$.
	\end{enumerate}
	\item Провести вычисления и построить графики для выборки из индивидуального варианта.
\end{itemize}


\chapter{Теория}
Пусть $\vec{x} = (x_1, \dots x_n)$ -- выборка из генеральной совокупности $X$ объема $n$.
\begin{enumerate}[wide=0pt]
	\item Вариационный ряд --- отсортированная выборка их генеральной совокупности $X$;
	
	\item Максимальное значение выборки: $M_{min} = x_{(1)},$ являющийся первым элементом вариационного ряда;
	
	\item Минимальное значение выборки: $M_{max} = x_{(n)},$ являющийся последним элементом вариационного ряда;
	
	\item Размах выборки: $R = M_{max} - M_{min},$
	
	\item Оценка математического ожидания: $\hat{\mu}(\vec{x}) = \cfrac{1}{n}\sum_{i = 1}^{n}x_i,$
	
	\item Оценка дисперсии: $S^2(\vec{x}) = \cfrac{1}{n}\sum_{i = 1}^{n}(x_i - \bar{x})^2.$
\end{enumerate}


$n(t, \vec{x})$ -- число компонент вектора $\vec{x}$, которые меньше чем $t$. \textit{Эмпирической функцией распределения}, построенной по выборке $\vec{x}$ называется функция $F_n: \mathbb{R} \rightarrow \mathbb{R}$, определенная правилом \ref{pdf}.

\begin{equation}\label{pdf}
	F_n(t) = \cfrac{n(t, \vec{x})}{n}
\end{equation}

\textit{Интервальный статистический ряд} --- это ряд $J = [x_{(i)}, x_{(n)}]$, 
который разбивают на $m$ промежутков, 
ширина которых определяется согласно \ref{delta}:

\begin{equation}\label{delta}
	\Delta = \cfrac{\left|J\right|}{m} = \cfrac{x_{(n)} - x_{(i)}}{m},
\end{equation}


\begin{equation}
	\begin{array}{ll}
		J_i = \left[x_{(1)} + \left(i - 1\right)\Delta; \; x_{(i)} + i\Delta\right), \; i = \overline{1,\: m - 1},\\
		J_m = \left[ x_{(1)} + \left(m - 1\Delta\right), \; x_{(n)}\right).
	\end{array}
\end{equation}

\textit{Эмпирической плотностью} распределения соответствующей выборке $\vec{x}$ называется функция \ref{cdf}:
\begin{equation}\label{cdf}
	f_n(x) = 
	\begin{cases}
		\cfrac{n_i}{n \cdot \Delta} & ,x \in J_i, \\
		0 & ,\text{иначе.}
	\end{cases}
\end{equation}

\chapter{Практика}

\begin{lstlisting}
function main()
	pkg load statistics
	
	X = dlmread("1.txt", ",");
	
	X = sort(X);
	% disp(X)
	
	% (a) минимальное и максимальное значение
	m_max = max(X);
	m_min = min(X);
	
	fprintf("(a) Максимальное значение выборки (M_max) = %f\n", m_max)
	fprintf("    Минимальное значение выборки  (M_min) = %f\n", m_min)
	fprintf("-------------------------------------------------\n")
	
	% (б) размах выборки
	r = m_max - m_min;
	fprintf("(б) Размах выборки (R) = %f\n", r)
	fprintf("------------------------------\n")
	
	% (в) вычисление оценок MX DX
	n = length(X);
	mu = sum(X) / n;
	s_2 = sum((X - mu).^2) / (n - 1);
	sigma = sqrt(s_2);
	
	fprintf("(в) Оценка математического ожидания (mu) = %f\n", mu)
	fprintf("    Оценка дисперсии (s_2) = %f\n", s_2)
	fprintf("----------------------------------------\n")
	
	% (г) группировка значений выборки в m = [log_2 n] + 2 интервала
	m = floor(log2(n)) + 2;
	
	bins = [];
	cur = m_min;
	
	for i = 1:(m + 1)
		bins(i) = cur;
		cur = cur + r / m;
	end
	
	eps = 1e-6;
	counts = [];
	j = 1;
	
	for i = 1:(m - 1)
		cur_count = 0;
		for j = 1:n
			if (bins(i) < X(j) || abs(bins(i)-X(j)) < eps) && X(j) < bins(i+1)
				cur_count = cur_count + 1;
			endif
		endfor
		counts(i) = cur_count;
	endfor
	
	cur_count = 0;
	
	for j = 1:n
		if (bins(m) < X(j) || abs(bins(m) - X(j)) < eps) && (X(j) < bins(m + 1) || abs(bins(m + 1) - X(j)) < eps)
			cur_count = cur_count + 1;
		endif
	endfor
	
	counts(m) = cur_count;
	
	fprintf("(г) группировка значений выборки в m = [log_2 n] + 2 интервала:\n");
	
	for i = 1:(m - 1)
		fprintf("    [%f : %f) - %d вхожд.\n", bins(i), bins(i + 1), counts(i));
	end
	
	fprintf("    [%f : %f] - %d вхожд.\n", bins(m), bins(m + 1), counts(m));
	fprintf("----------------------------------------\n");
	
	% (д)  построение гистограммы и графика функции плотност
	
	fprintf("(д) построение гистограммы и графика функции плотности\n");
	fprintf("    распределения вероятностей нормальной случайной величины\n");
	
	figure;
	hold on;
	grid on;
	n = length(X);
	delta = r / m;
	middles = zeros(1, m);
	xx = zeros(1, m);
	
	for i = 1:m
		xx(i) = counts(i) / (n * delta);
	endfor
	
	for i = 1:m
		middles(i) = bins(i + 1) - (delta / 2);
	endfor
	
	fprintf("    высоты столбцов гистограммы:\n");
	
	for i = 1:m
		fprintf("    [%d] : %f\n", i, xx(i));
	endfor
	
	fprintf("[проверка] площадь гистограммы s = %f\n", sum(xx) * delta);
	
	set(gca, "xtick", bins);
	set(gca, "ytick", xx);
	set(gca, "xlim", [min(bins) - 1, max(bins) + 1]);
	bar(middles, xx, 1, "facecolor", "g", "edgecolor", "w");
	
	X_n = m_min:(sigma / 100):m_max;
	X_pdf = normpdf(X_n, mu, sigma);
	plot(X_n, X_pdf, "r");
	xlabel('X')
	ylabel('P')
	print -djpg hist.jpg
	hold off;
	
	fprintf("----------------------------------------\n");
	
	% (е) построение графика эмпирической функции распределения
	
	fprintf("(е)построение графика эмпирической функции распределения\n");
	fprintf("   и функции распределения нормальной случайной величины\n");
	
	figure;
	hold on;
	grid on;
	n = length(X);
	xx = zeros(1, length(X));
	curss = 1;
	
	bins = zeros(1, length(X));
	bins(1) = X(1);
	for i = 2:length(X)
		if (bins(curss) != X(i))
			curss+= 1;
			bins(curss) = X(i);
			xx(curss) = xx(curss-1);
		endif
		if (bins(curss) == X(i))
			xx(curss) += 1;
		endif
	end
	
	xx = xx ./ length(X);
	
	for i = curss:length(X)
		xx(i) = 1;
	end
	
	X_n = (min(X) - 0.5):(sigma / 100):(max(X) + 1.5);
	X_cdf = normcdf(X_n, mu, sigma);
	plot(X_n, X_cdf, "r");

	stairs(bins, xx);
	xlabel('X')
	ylabel('F')
	print -djpg cdf.jpg
	hold off;
end
	
\end{lstlisting}



\chapter{Результаты}

\img{160mm}{result}{Результат работы программы}

\img{120mm}{hist}{Гистограмма и график функции плотности распределения вероятностей нормальной случайной величины}

\img{120mm}{cdf}{График эмпирической функции распределения и функции распределения нормальной случайной величины}
